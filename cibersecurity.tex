\documentclass{article}
\usepackage{graphicx} % Required for inserting images
\usepackage[utf8]{inputenc}
\usepackage[spanish]{babel}
\usepackage{xcolor}
\usepackage{amsmath}
\usepackage{amssymb}
\usepackage{titling}

\setlength{\parindent}{0pt}
\pagecolor{black}
\color{white}
\title{
	% \includegraphics[width=10cm]{imgs/portada.jpg}
	\textbf{Ciberseguridad avanzada}
}
\author{BlasAST}
\date{May 2025}

\begin{document}
\maketitle

\newpage

\tableofcontents

\newpage
\section{Introducción}

La ciberseguridad es una disciplina fundamental en esta era digital, las
amenazas cibernéticas crecen en complejidad y frecuencia a la vez que las
organizaciones se vuelven cada vez más dependientes de la tecnología para
sus tareas diarias. Proteger la información y los sistemas se convierte
en una prioridad importantisima. Para entender la ciberseguridad se necesita
un conocimiento sólido de los fundamentos que la conforman, incluyendo
modelos organizativos, conceptos básicos y tecnológicos al igual que el rol
que desempeñan las personas en este ámbito.

\begin{enumerate}
	\item Los \b{modelos organizativos en ciberseguridad} proporcionan una
		estructura para implementar gestionar y superviar las estrategias de
		protección de la información de una organización.
	\item Los conceptos basicos y tecnológicos son sobre los que se construye
		todo el sistema de protección incluyendo los conocimientos de los tipos
		de amenazas que hay.
	\item El rol de las personas es importante dado que son una parte vital de
		la defensa dado que son el eslabon más debil y más fuerte de la cadena.
		Hay que concienciar a \b{todos} los empleados para que no comprometan
		información de la organización.
\end{enumerate}

Las amenazas cibernéticas evolucionan constantemente por lo que los conocimientos
y habilidades deben de actualizarse continuamente. TODO debe funcionar en armonía
con los demás componentes de para proteger lo máximo posible.

\section{Modelos organizativos}

Estructuras que estratégicas que determinan cómo una organización gestiona, aborda
y protege sus activos digitales frente a amenazas cibernéticas. Es imprescindible
para garantizar la seguridad y la resiliencia frente a posibles ataques.
Para que un modelo organizativo sea robusto debe de comenzar con una clara
definición de roles y responsabilidades en la organización dejando claro quien
es el responsable de la seguridad en distintos niveles y que entienda que
sus miembros entiendan su papel.

Debe de haber políticas claras y procedimientos estandares que guíen las acciones
de la organización en diversas situaciones. Desde política de contraseñas, manejo
de datos sensibles, respuestas a incidentes de seguridad y formación continua
de los empleados. Frente a un incidente cibernético debe de darse una respuesta
rapida coherente y efectiva, minimizando así el impacto en la organización.

Estos modelos organizativos deben de ser también lo suficientemente flexibles para
adaptarse a las necesidades de la organización y a las distintas amenazas en
evolución. Por ello, las organizaciones debe de estar dispuestas a revisar y ajustar
sus modelos a medida que sugen nuevas tecnologías y amenazas.

Con un modelo organizativo bien diseñado conseguimos una defensa solida.\\

\qquad Tipos de modelos:

\begin{itemize}
	\item Centralizado: (todas las decisiones y funciones de seguridad se concentran
		en un solo equipo o departamento, normalmente "CISO - Chief Information
		Security Officer") manejando politicas, procedimientos y controles de
		seguridad.
\end{itemize}

Ventajas:

\begin{enumerate}
	\item Consistencia en la aplicación de las políticas de seguridad.
	\item Mejor coordinación y control centralizado de los recursos de seguridad.
	\item Facilidad de implementar estándares uniformes en toda la organización.
\end{enumerate}

Desventajas:

\begin{enumerate}
	\item Puede ser menos ágil al responder a incidentes locales o específicos
	\item Riesgo de sobrecargar el equipo central con todas las responsabilidades
	\item La falta de autonomía en las unidades locales puede limitar la adaptabilidad
\end{enumerate}

\begin{itemize}
	\item Descentralizado: (Las responsabilidadesse distribuyen en diferentes
	departamentos o unidades de negocio cada una con su equipo y toman decisiones
	de manera independiente bajo las directrices generales de la organización)
\end{itemize}

Ventajas:

\begin{enumerate}
	\item Mayor flexibilidad y rapidez para responder a amenazas locales o específicas
	\item Adaptación de las políticas de seguridad a las necesidades de cada unidad
	\item Fomenta la autonomíay la responsabilidad en distintas áreas de la organización
\end{enumerate}

Desventajas:

\begin{enumerate}
	\item Riesgo de inconsistencias al aplicar las políticas de seguridad
	\item Posible generación de duplicación de esfuerzos y recursos
	\item Dificultad en la coordinación y comunicación entre equipos de seguridad
\end{enumerate}



\section{Conveptos básicos y tecnológicos}

La ciberseguridad abarca una amplia gama de conceptos básicos y tecnológicos
para proteger la integridad, confidencialidad y disponibilidad de la información
en un entorno digital.

Con el aumento de las amenazas cibernéticas conocer sobre esto es fundamental 
para organizaciones de todos los tamaños y sectores.

A continuación 
\end{document}